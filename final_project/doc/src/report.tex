\documentclass{myreport}

\begin{document}

\maketitle

% 目录
\newpage
\tableofcontents
\newpage

% 正文
\section{系统需求分析}
  \subsection{需求概述}
    对于排课管理系统, 课程设计的要求如下:
    \begin{itemize}
      \item 实现班级, 课程等基本信息的管理;
      \item 实现学生, 教师信息的管理;
      \item 实现班级课程及课程的任课教师和排课管理;
      \item 创建存储过程检测指定教师, 指定节次是否有课;
      \item 创建存储过程生成指定班级的课程表;
      \item 创建存储过程生成指定老师的课程表;
      \item 建立数据库相关表之间的参照完整性约束.
    \end{itemize}

    即通过数据库自动排课并提供给学生查询,让学生和老师可以查询具体时间安排.
    该系统可以通过以下实体集实现
    \begin{itemize}
      \item \emph{教学楼}实体集, 包含楼号和楼名;
      \item \emph{教室}实体集, 包含楼号,教室号和容量;
      \item \emph{院系}实体集, 包含院系编号和院系名;
      \item \emph{课程}实体集, 包含课程号, 课程名, 课程类型, 开课学院;
      \item \emph{教师}实体集, 包含教师的编号, 姓名,院系, 职称, 研究方向\footnote{可能是老师工作的具体院系, 如"计算机系", 也可能是其他研究所,如"基础数学研究所"};
      \item \emph{班级}实体集, 班级ID, 班级名, 人数, 所属院系;

    \end{itemize}

  \subsection{组织结构分析}
    本系统适用于教师与学生对课程的管理, 提供给学生,教师所有表的查看权限, 数据库管理员拥有其他权限.

  \subsection{管理功能分析}
    排课是个综合系统,需要从教务系统中导入数据(或者由数据库管理员人工导入),实现课程安排,即课程管理, 同时将课程的信息分别存储汇总, 部分实现教师管理, 时间管理,教室管理, 班级管理.
    % \cref{fig:function}
    \begin{figure}[H]
      \centering
      \includegraphics[width = 0.6\textwidth]{function}
      \caption{排课系统的管理功能}
      \label{fig:function}
    \end{figure}

  \subsection{业务流分析}

  \subsection{数据字典}
    根据数据流图中所涉及的信息,并对信息进行相应的分析,确定出所有数据项的描述内容,其主要分为数据项名称、类型、长度和取值范围,如\cref{t:data_dict}所示

    % TODO
    % http://sparkandshine.net/latex-use-notes-longtable-with-examples/
    \begin{longtabu} to \textwidth {clcXXX}
      \caption{数据字典(具体的数据的大小参考\cite{tinyint})}
      \label{t:data_dict} \\
      \toprule[1.5pt]
        \makebox[0.1\textwidth]{名称}  &
        \makebox[0.2\textwidth]{含义} &
        \makebox[0.1\textwidth]{类型} &
        \makebox[0.1\textwidth]{大小} &
        \makebox[0.1\textwidth]{取值范围} &
        \makebox[0.2\textwidth]{备注} \\
      \midrule[1pt]
      \endhead

      \bottomrule[1.5pt]
      \endfoot


      % 表格正文
        楼号    & 教学楼的编号 & tinyint & \SI{1}{B} & 0-255 & \\
        楼名    & 教学楼的名称 & char(5) & \SI{15}{B} & 长度$\le 5$ & \\
        容量    & 教学楼的容量 & tinyint & \SI{1}{B} & 0-255 & \\
        院系编号 & 院系的编号 & tinyint & \SI{1}{B} & 0-255 & \\
        院系名   & 院系名 & char(8) & $\le\SI{24}{B}$ & 0-255 & \\
        课程号   & 课程编号 & char(20) & \SI{20}{B} & 20位 & \\
        课程名   & 课程名 & char(10) & $\le \SI{30}{B}$ & 10位 & \\
        类型名   & 课程的类型 & char(10) & $\le \SI{30}{B}$ & \\
        教师号   & 教师编号 & char(20) & \SI{20}{B} & 20位 & \\
        教师名   & 教师的姓名 & char(10) & $\le \SI{30}{B}$ & \\
        职称     & 教师的职称 & char(3) &\SI{9}{B}  & 助教, 讲师, 副教授, 教授 \\
        班号     & 班级编号 & char(20) & \SI{20}{B} & 20位 & \\
        班名     & 班级的全名 & char(10) & $\le \SI{30}{B}$ & \\
        人数     & 班级人数 & tinyint & \SI{1}{B} & 0-255 & \\
        时间号   & 上课时间的标识 & char(20) & \SI{20}{B} & 20位 & \\
        日       & 星期几 & tinyint & \SI{1}{B} & 0-255 & 星期用1-5代替\\
        开始时间 & 上课时间 & tinyint & \SI{1}{B} & 0-255 & 假设上课时间都是整点, 24小时制\\
        结束时间 & 下课时间 & tinyint & \SI{1}{B} & 0-255 & 假设下课时间都是整点, 24小时制\\
        开始周   & 第几周开始 & tinyint & \SI{1}{B} & 0-255 & \\
        结束周   & 第几周结束 & tinyint & \SI{1}{B} & 0-255 & \\
        is\_odd  & 单周上课 & bit & \SI{1}{B} & 0,1 & 默认为1 \\
        is\_even & 双周上课 & bit & \SI{1}{B} & 0,1 & 默认为1 \\
        节号      & 上课节的标识 & char(20) & \SI{20}{B} & 20位 & \\
        学期      & 在上学期或下学期 & bit & \SI{1}{B} & 0,1 & 每年第一学期为0,第二学期为1 \\
        年        & 年份 & smallint & \SI{2}{B} &$-32,768$- $32,767$ & \\

        % \specialrule{0em{8pt}{1pt}}
    \end{longtabu}




\section{数据库概念结构设计}

\section{数据库逻辑结构设计}

\section{数据库的物理实现}

\section{数据库功能调试}
(包括视图、索引等内容的测试)

\section{应用程序设计}

\section{设计总结}





% 参考文献
\nocite{silberschatz1997database} % 数据库系统概论
\nocite{sqldbm} % 数据库画图软件
\nocite{pyqt5} % PyQt5 参考文档
\bibliography{reference} % 提示从reference数据库获取文献信息,来打印参考文献


\end{document}


% 命令用法
% % 插入代码
% \begin{lstlisting}[language=sql]
%   use university;
%   select * from university;
% \end{lstlisting}
