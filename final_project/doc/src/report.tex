\documentclass{myreport}

\begin{document}

\maketitle

% 目录
\newpage
\tableofcontents
\newpage

% 正文
\section{系统需求分析}
  \subsection{需求概述}
    对于排课管理系统, 课程设计的要求如下:
    \begin{itemize}
      \item 实现班级, 课程等基本信息的管理;
      \item 实现学生, 教师信息的管理;
      \item 实现班级课程及课程的任课教师和排课管理;
      \item 创建存储过程检测指定教师, 指定节次是否有课;
      \item 创建存储过程生成指定班级的课程表;
      \item 创建存储过程生成指定老师的课程表;
      \item 建立数据库相关表之间的参照完整性约束.
    \end{itemize}

    即通过数据库自动排课并提供给学生查询,让学生和老师可以查询具体时间安排.
    该系统可以通过以下实体集实现
    \begin{itemize}
      \item "教师"实体集, 包含教师的编号, 姓名,院系, 职称, 研究方向\footnote{可能是老师工作的具体院系, 如"计算机系", 也可能是其他研究所,如"基础数学研究所"};
      \item "学生"实体集, 包含学生的学号, 姓名, 班级, 院系等信息;
      \item "教室"实体集, 包含楼号, 教室号和容量;
      \item "院系"实体集, 包含院系编号和院系名;

    \end{itemize}

  \subsection{组织结构分析}
        

\section{数据库概念结构设计}

\section{数据库逻辑结构设计}

\section{数据库的物理实现}

\section{数据库功能调试}
(包括视图、索引等内容的测试)

\section{应用程序设计}

\section{设计总结}





% 参考文献
\nocite{silberschatz1997database} % 数据库系统概论
\nocite{sqldbm} % 数据库画图软件
\nocite{pyqt5} % PyQt5 参考文档
\nocite{tinyint} % tiny int 字节大小
\bibliography{reference} % 提示从reference数据库获取文献信息,来打印参考文献


\end{document}


% 命令用法
% % 插入代码
% \begin{lstlisting}[language=sql]
%   use university;
%   select * from university;
% \end{lstlisting}
