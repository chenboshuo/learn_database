\documentclass{myreport}

\begin{document}

\maketitle

% 目录
\newpage
\tableofcontents
\newpage

% 正文
\section{系统需求分析}
  \subsection{需求概述}
    对于排课管理系统, 课程设计的要求如下:
    \begin{itemize}
      \item 实现班级, 课程等基本信息的管理;
      \item 实现学生, 教师信息的管理;
      \item 实现班级课程及课程的任课教师和排课管理;
      \item 创建存储过程检测指定教师, 指定节次是否有课;
      \item 创建存储过程生成指定班级的课程表;
      \item 创建存储过程生成指定老师的课程表;
      \item 建立数据库相关表之间的参照完整性约束.
    \end{itemize}

    即通过数据库自动排课并提供给学生查询,让学生和老师可以查询具体时间安排.
    改系统可以通过以下实体集实现
    \begin{itemize}
      \item "教师"实体集, 包含教师的编号, 姓名, 部门, 院系, 职称等信息;
      \item "学生"实体集, 包含学生的学号, 姓名, 班级, 院系等信息;
      \item "课程"实体集, 包含课程号,课程类型(是否是必修课), 班级,单双周, 开课周, 节数\footnote{即每一节大课有多少小节, 大多数课程都是两节, 有部分课程是三节或一节}.
    \end{itemize}

    % % 插入代码
    % \begin{lstlisting}[language=sql]
    %   use university;
    %   select * from university;
    % \end{lstlisting}

% 参考文献
\nocite{silberschatz1997database} % 数据库系统概论
\nocite{sqldbm} % 数据库画图软件
\bibliography{reference} % 提示从reference数据库获取文献信息,来打印参考文献

\end{document}
