\documentclass[UTF8]{ctexart}

% 参考文献
\bibliographystyle{plain} % 声明参考文献格式
\usepackage{url}


% 设置页边距
\usepackage{geometry}
\geometry{a4paper,scale=0.8}

% 插入PDF
\usepackage{pdfpages}

% 插图功能的相关宏包
\usepackage{graphicx}
\usepackage{float}
\graphicspath{{figures/}}

% % 修改字体
% \usepackage{type1cm}  %(其中的 cm 为 Computer Modern 的缩写)
% \CJKfamily{song}  % 宋体

% 增加目录中的项目用tocbibind
\usepackage[nottoc]{tocbibind}
% 宏会默认增加目录本身,参考文献,索引等项目.noctoc取消了目录中显示目录
\begin{document}

\includepdf[pages={1}]{figures/cover.pdf}

% 目录
\newpage
\tableofcontents
\newpage

% 正文
\section{系统需求分析}
  \subsection{需求概述}
    对于排课管理系统, 课程设计的要求如下:
    \begin{itemize}
      \item 实现班级, 课程等基本信息的管理;
      \item 实现学生, 教师信息的管理;
      \item 实现班级课程及课程的任课教师和排课管理;
      \item 创建存储过程检测指定教师, 指定节次是否有课;
      \item 创建存储过程生成指定班级的课程表;
      \item 创建存储过程生成指定老师的课程表;
      \item 建立数据库相关表之间的参照完整性约束.
    \end{itemize}

    即通过数据库自动排课并提供给学生查询,让学生和老师可以查询具体时间安排.
    改系统可以通过以下实体集实现
    \begin{itemize}
      \item "教师"实体集, 包含教师的编号, 姓名, 部门, 院系, 职称等信息;
      \item "学生"实体集, 包含学生的学号, 姓名, 班级, 院系等信息;
      \item "课程"实体集, 包含课程号,课程类型(是否是必修课), 班级,单双周, 开课周, 节数\footnote{即每一节大课有多少小节, 大多数课程都是两节, 有部分课程是三节或一节}.
    \end{itemize}



% 参考文献
\nocite{silberschatz1997database} % 数据库系统概论
\nocite{sqldbm} % 数据库画图软件
\bibliography{reference} % 提示从reference数据库获取文献信息,来打印参考文献

\end{document}
